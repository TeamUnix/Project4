\section{Time box 3}
\subsection{Time box planning}
Overview of the what work is put into which time boxes.
\begin{figure}[H]
	\begin{centering}
		 %\includegraphics[width=1.0\textwidth]{images/tb_r2.pdf}
		%\caption{Updated time-box}
		\missingfigure{Updated timeline}
	\end{centering}
\end{figure}

\subsubsection{Work to be done in this time box}
\begin{itemize}
	\item Device driver
	\begin{itemize}
		\item Create a device driver framework from the scull example and the device drivers included in the distribution. 
		\item Convert the ADC driver into a device driver.
	\end{itemize}
	\item Power switch
	\begin{itemize}
		\item Schematic
		\item Printed circuit board
		\item Mount component
		\item Test board
	\end{itemize}
	\item Host controller
	\begin{itemize}
		\item Master wishbone
		\item CPU interface
	\end{itemize}
\end{itemize}
\paragraph{Description:}
\begin{description}
	\item[Power switch] A single switch port board for the power switch has to be made, as an essential part for the power switch
	\item[Host controller] is made in the FPGA, and is responsible for the communication between the Spartan6 and the ARM7, it shall be made with wishbone master interface 
\end{description}

\subsection{Device driver skeleton - Dennis}
The goal is to create a simple framework for creating device drivers for the LPC2478 platform. To do so, the scull example provided is used together with sample code provided with the uClinux distribution. The framework is tested by converting an ADC driver (running on the bare metal) into a device driver. Down below code snippets from the ADC is extracted an commented. 
\p The module registration is found in the \textit{file\_operations} structure:
\begin{lstlisting}[language=c]
static struct file_operations adc_fops = {
  .owner   = THIS_MODULE,
  .read    = adc_read,
  .open    = adc_open,
  //.write   = none,
  .release = adc_close,
};
\end{lstlisting}

The structure contains pointer values to the given functions, where these functions can be accessed from user space by:
\begin{table}[H]
	\centering
	\begin{tabular}{|l|l|l|}
		\hline
			Events		& User Functions & Kernel Functions \\ \hline
			Load Module 	& insmod	& module\_init()		 \\ 
			Open		& fopen	& file\_operations: open	 \\ 
			Read		& fread	& file\_operations: read	 \\ 
			Write			& fwrite 	& file\_operations: write	 \\ 
			Close		& fclose 	& file\_operations: release\\ 
			Remove		& rmmod 	& module\_exit()		 \\
		\hline
	\end{tabular}
	\caption{Function event table.}
\end{table}
Loading and removing a module is not included in the file operation structure, but declared aside as pointers to the two functions used:
\begin{lstlisting}[language=c]
// Define init and exit functions
module_init(adc_mod_init);
module_exit(adc_mod_exit);
\end{lstlisting}

In the init function several different things are done:
\begin{itemize}
	\item Setting up the registers (copy from the \textit{bare metal} driver file).
	\item Registration of the device (no need for dynamic numbering, so the major number is a static number not used by other drivers.)
	\item Allocating the char driver and adding it to the environment.
\end{itemize}

Code pieces from setting up the ADC:

\begin{lstlisting}[language=c]
//Register the device
	dev = MKDEV(adc_major, adc_minor);
	result = register_chrdev_region(dev, NUM_ADC_DEVICES, "adc");
//Allocate and add the adc device.
	adc_cdev = cdev_alloc();
	cdev_init(adc_cdev, &adc_fops);
	adc_cdev->owner = THIS_MODULE;
	adc_cdev->ops   = &adc_fops;

	result = cdev_add(adc_cdev, dev, NUM_ADC_DEVICES);
	
static void adcInit(void){
	volatile u32 tmp = 0;
	m_reg_bfs(PCONP, (1 << 12)); //Power on ADC
	m_reg_write(AD0CR, 0);
	m_reg_write(AD0CR,
			(4<<8)			|	//set clock division factor
			(0<<17)			|	//CLKS  = 0, 11 clocks = 10-bit result
			(1<<21)			|	//PDN   = 1, ADC is active
			(1<<24));			//start a conversion
	}
\end{lstlisting}
The exit function undoes the steps done in the initialization function.
\p Before reading the ADC channels, the channel shall be opened for reading. In the ADC code the pins are pin function are setup to be of type ADC. As the pins belongs to different PINSEL registers, a if else statement is used to set the right registers.
\begin{lstlisting}[language=c]
// inside the function: 
static int adc_open(struct inode* inode,  struct file* file){

// Check if the channel has been verified
if(chRefCnt[channel] == 0){
	file->private_data = (void *) channel;
	if(channel >= 0 && channel <=3)
		m_reg_bfs(PINSEL1, enable_bits[channel]);
	else if(channel > 3 && channel < 6)
		m_reg_bfs(PINSEL3, enable_bits[channel]);
	else if(channel >= 6 && channel < 8)
		m_reg_bfs(PINSEL0, enable_bits[channel]);
}
chRefCnt[channel]++;
\end{lstlisting}
When finish reading, the file is closed again and the opposite operation is performed (deselect the ADC function for the proper IO pin).
\p When reading the ADC, the result retrieved is converted from decimal into a string before it is returned to the user. Reading and writing to and from the CPU registers is done through the kernel space functions \textit{m\_reg\_read} and \textit{m\_reg\_write}. So the register writes and reads in the driver file for the ADC is performed as parameters sent to these functions:
\begin{lstlisting}[language=c]
//start conversion now (for selected channel)
m_reg_write(AD0CR, ((1 << channel) | (1 << 24)));

//wait til done
while ((m_reg_read(AD0GDR) & (1<<31)) == 0);

//get result from global register and adjust to 10-bit integer
result = (m_reg_read(AD0GDR) >> 6) & 0x3FF;
\end{lstlisting}

\subsubsection{Testing the driver - Dennis \& Paulo}
To proper test the driver, a small user space program has been written. The program opens the device, reads it and closes it again.
The augment sent to the program selects which channel should be read (ad0, ad1, ad2 etc.).

\begin{lstlisting}[language=c]
#define ADC "/dev/adc"
	
	//Format string to "Analog value"
	//Ex 1023 -> 3.30
void stringFormat(int value){
	char str[10];
	if((value%100)>9)
		sprintf(str, "%i.%i",value/100,value%100);
 	else
 		sprintf(str, "%i.0%i",value/100,value%100);
 		
	printf("%s\n",str);
}

int main(int argc, char *argv[]) {
	FILE *fp;
	char read[10];

	if ((fp = fopen(strcat(ADC,argv[1]),"r"))==NULL){
		printf("Cannot open file.\n");
		exit(0);
	}
	
	fread(read,1,10,fp);
	stringFormat(atoi(read)*330/1023);
	fclose(fp);
	return 0;
}
\end{lstlisting}



\subsection{Power switch - Paulo}

\subsection{Host coWriting device drivers in Linux: A brief tutorialntroller - Theis}
The host controller is a part in the Spartan6 that controls the parts in the Spartan6 with command from the ARM7 that have connection to the host controller block.
\begin{figure}[H]
	\begin{centering}
		 \includegraphics[width=1.0\textwidth]{images/host_controller.pdf}
		\caption{Host controller}
		%\missingfigure{host controller diagram}
	\end{centering}
\end{figure}
\subsubsection{External memory controller}
The EMC is only used in static mode
\subsubsection{CPU interface}
The CPU interface communicate with the EMC on one site and with the master wishbone on the other site. The purpose of this interface is to determine if the EMC wants to communicate with the Spartan6, and if the ARM7 wants to read or write data.

\begin{lstlisting}[language=VHDL]
process (Clk,Rst)
	begin  
		if Rst = '1' then								--Reset set everything to 0
			Wr_o	<= '0';
			Rd_o	<= '0';
			A_o		<= (others => '0');
			D_o		<= (others => '0');
			CpuD_o	<= (others => '0');
		elsif (Clk'event and Clk = '1') then
			if CpuCs_i = '1' then					--Check chip select
				A_o	<= CpuA_i;							--Adress routing
					if CpuRd_i = '1' then			--Reading
						Rd_o	<= CpuRd_i;
						Wr_o	<= CpuWr_i;
						CpuD_o	<= D_i;					--Wishbone data out to Cpu data input
					elsif CpuWr_i = '1' then	--Writing
						Wr_o	<= CpuWr_i;					
						Rd_o	<= CpuRd_i;
						D_o		<= CpuD_i;				--Cpu data output to wishbone data input
					else
						Wr_o	<= CpuWr_i;		
						Rd_o	<= CpuRd_i;
						D_o		<= CpuD_i;
						CpuD_o	<= (others => '0');
					end if;
			else													--If chip select not high everything is set to 0
				Wr_o	<= '0';	
				Rd_o	<= '0';
				D_o		<= (others => '0');
				A_o		<= (others => '0');
				CpuD_o	<= (others => '0');
			end if;
		end if;
end process;
\end{lstlisting}
The block is synchronous with the clock and have an asynchronous reset.If the reset is pressed everything is set to zero to tell the master wishbone to not do anything. When the chip select is high, it tells the block that the ARM7 want to communicate with the Spartan6. After chip select i activated, the block check if the ARM7 want to read or write data from or to the Spartan6. If the ARM7 wants to read data, the interface block route the data from master wishbone to the data vector on the EMC. Else if the ARM7 wants to write data, the data vector on the EMC is routed to the data input for the master wishbone. In any case of read or write the read and write output from the EMC is routed to the master wishbone. The address input from the EMC is routed as soon as the chip select is activated. If the chip select is not active everything is routed to zero, to secure that the master wishbone do not write any data. The data output to the EMC is tri-stated in the wrapper code if the chip select is not active, to not disturb other communication on the EMC bus.

\subsubsection{Wishbone}

Wishbone is a computer bus for integrated circuit communication. It sets up some standard communication rules to use when designing IP cores. This make it easy to reuse code on different hardware and in different systems. The wishbone bus is a logic bus, this means that there is no rules for voltage levels, it only works with ones and zeros. In this project the wishbone is used as illustrated on the picture below.
\begin{figure}[H]
	\begin{centering}
		 \includegraphics[width=0.8\textwidth]{images/typical_usage.png}
		\caption{Wishbone connection for the host controller}
		%\missingfigure{host controller diagram}
	\end{centering}
\end{figure}
The EMC on the ARM7 is controlling the wishbone master, witch is controlling all the wishbone slaves that is in the system. This make it easier to write a driver for the ARM7 that controls the master wishbone through the EMC. Then the master wishbone control every IP core with at wishbone slave interface connected. In this project the wishbone is going to use single read and write cycles, the single read cycle timing diagram is shown below.
\begin{figure}[H]
	\begin{centering}
		 \includegraphics[width=0.65\textwidth]{images/wb_single_read.png}
		\caption{Single read cycle}
		%\missingfigure{host controller diagram}
	\end{centering}
\end{figure}
When a single read cycle is started, the master wishbone presents a valid address, on the address register, it set the write enable to zero to show that this is a read cycle, the SEL\_O is set, to show where on the data input it expect the data to be read, it also sets the CYC\_O and STB\_O high to indicate the start of a cycle and a phase.


On the next raising clock edge, the slave decode the input and set the acknowledge bit high in response to the SEL\_O. It also presents valid data on the slave data output. The master monitors the acknowledge and prepares to latch data.


On the third raising clock edge the master wishbone latches the data, and set the STB\_O and CYC\_O to zero to end the cycle and the phase, the slave set the acknowledge bit low again as response to STB\_O.\\
The timing diagram for the single write cycle is shown below.
\begin{figure}[H]
	\begin{centering}
		 \includegraphics[width=0.65\textwidth]{images/wb_single_write.png}
		\caption{Single write cycle}
		%\missingfigure{host controller diagram}
	\end{centering}
\end{figure}
When the master wishbone wants to write to the slave, it present valid address and data on the address register and data output, it also sets the write enable high to indicate the cycle is write. As with the read the STB\_O and CYC\_O is set high to indicate cycle and phase start, all this is done on the first raising clock edge.

On the second raising clock edge the slave decode the input and set the acknowledge bit high in response to the SEL\_O. It also prepares to latch data from the master. The master monitors the acknowledge bit and prepare to terminate the cycle.

On the last raising clock edge the slave latches the data, the master set the STB\_O and CYC\_O to zero to end the cycle and the phase, the slave set the acknowledge bit low again as response to STB\_O.

