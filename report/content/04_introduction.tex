\chapter{Introduction}
This report is divided into three parts. The first part introduce the problem statement. The second part concerns the method that has been used for the working process, and the results for the problem that has been developed through the process. The last part recapitulates the entire project, and verifies if the requirements and the problems, defined in the problem statement are complied with.
The project concerns an energy system for handling energy production and consumption, the project is divided into smaller parts where this report is about the energy hub, which is handling inputs from other devices that produces or consumes energy.
\paragraph{Reading guidelines}

\begin{table}
    \begin{tabular}{|l|l|}
        \hline
        \textbf{Abbreviation} & \textbf{Meaning} \\ \hline
        ARM		& Advanced RISC Machine \\ \hline
        BFM		& Bus Functional Module \\ \hline
        EMC		& External Memory Controller\footnote{A peripheral in the LPC2478} \\ \hline
        EMC		& Electromagnetic Compatibility\footnote{} \\ \hline
        FPGA	& Field Programmable Gate Array \\ \hline
        PLC		& Power Line Communication \\ \hline
        PS		& Power Switch \\ \hline
        RISC	& Reduced Instruction Set Computing \\
        \hline
    \end{tabular}
\end{table}
%How to read this report (Reading guidelines)
%How was the project initiated?
%What ideas, interests and thoughts are behind your choice of subject?
%Have others worked on the problem and what did they do?
%The introduction may include artefacts from the PreProject, e.g. Rich Picture. It may fill two pages and cover the problem widely
%It may be a good idea to write this part at the end of a project period.