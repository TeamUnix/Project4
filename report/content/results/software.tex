\chapter{Software design}
\section{Kernel setup}
\writer{Dennis Madsen}
%
The kernel does not have a MMU (memory management unit) implemented, which handles access to memory requested by the processor. Therefore memory addresses can be accessed directly from the processor without having to call certain kernel functions. This approach have been used in some of the code written but is however not a very generic way to write code. Therefore the drivers made cannot be copied directly to a distribution using a MMU.
\p The kernel has been setup and compiled according to Klaus Kolles wiki\footnote{\url{http://klaus.ede.hih.au.dk/index.php/Build_uClinux_for_EA2478_Board}}.

\section{Das U-Boot}
\writer{Dennis Madsen}
%
Das U-Boot is a universal boot loader with support for several micro controller architectures (including the LPC2478 processor).
\p According to the \textit{External memory controller} section in \textit{Time Box 7}, the external memory controller 2 has been setup in order to be able to send and receive commands to and from the FPGA device. 
\p The default startup text has been changed slightly in order to show that it is not the standard compiled u-boot boot loader the system makes use of. 

%%%%%%%%%%%%%%%%DEVICE DRIVERS%%%%%%%%%%%%%%%%%%%%
\section{Device drivers}
\writer{Dennis Madsen}
%
A device driver is a piece of software in the kernel which makes it possible for user space application to communicate with the hardware. For the project, 3 device drivers have been made.
\subsection{ADC}
The driver made by Embedded Artist (included in the kernel source tree), has been modified in order to be able to read all 8 ADC inputs. The documentation for modifying the ADC driver is found in \textit{Time box 3}. 
\p To access the ADC channels, the ADCx files in \textit{/dev/} can be read or written (where x goes from 0 to 7).
\subsection{UART}
Time box 6 describes the design of the UART driver. When issuing the help command the different setup options are returned by the driver.
\begin{lstlisting}[language=c, caption=UART help command]
# ./uart help
Available commands:
  IOCTL_HELP : show different help commands
  IOCTL_DEFAULT : 8N1, 9600 baud
  IOCTL_BAUD : send argument between 2400 and 230400
  IOCTL_WORDLEN : send argument 5-8 word lenght
  IOCTL_STOPBIT : send argument, 1 or 2 stop bits
  IOCTL_PARBIT : send arg, 0, 1(odd), 2(even), 3(Forced 1 stick), 4(Forced 0 stick
  IOCTL_FIFO : send arg 0 (off), 1 (on)
  IOCTL_FIFO_TRIG : send arg number to trig, 1, 4, 8 or 14 characters
\end{lstlisting}
Beside the control setup option, the driver is able to send or receive characters through the 4 on board UART's which can be accessed as UART0, UART1, UART2 and UART3 in the \textit{/dev/} folder in the distributions file system.
\subsection{GPIO}
The GPIO device driver is specially designed for the Energy HUB project, table \ref{tab:gpio_table} shows the different pins connected to the GPIO files found in \textit{/dev/} including their functionality.
\begin{table}[H]
	\centering
	\begin{tabular}{|p{2cm}|p{2cm}|p{3cm}|p{3cm}|}\hline
		GPIO nr.		& Pin nr.		& Name			& Direction		\\\hline
		GPIO0		& 0.12		& PLC\_nReset		& Output			\\\hline
		GPIO1		& 0.13		& PLC\_wake		& Output			\\\hline
		GPIO2		& 0.18		& PLC\_nSleep		& Output			\\\hline
		GPIO3		& 2.10		& FPGA\_int		& Input Interrupt	\\\hline
		GPIO4		& 1.6			& PS\_main		& Output			\\\hline
		GPIO5		& 1.23		& PS1\_in			& Output			\\\hline
		GPIO6		& 1.25		& PS2\_in			& Output			\\\hline
		GPIO7		& 1.19		& PS1\_out		& Output			\\\hline
		GPIO8		& 1.21		& PS2\_out		& Output			\\\hline
	\end{tabular}
	\caption{GPIO connection table}
	\label{tab:gpio_table}
\end{table}
\begin{itemize}
	\item PLC\_x pins controls the Power Line Module
	\item FPGA\_int is interrupt input from the FPGA signaling that data is ready to be read.
	\item PSx pins controls 3 power switch modules.
\end{itemize}
Documentation of the GPIO device driver can be found in \textit{Time Box 7}
%%%%%%%%%%%%%%%%%%USER SPACE%%%%%%%%%%%%%%%%%%%%%
\section{User space applications}
\writer{Dennis Madsen}
An user space application has been written for each device driver in order to simplify hardware calls from the systems main program. The implementation of each user space application is described in the corresponding device driver time box. 
\subsection{ADC}
The ADC program takes two arguments: number and reference. \textit{Number} indicates the ADCx that should be read where \textit{reference} defines how the read value should be converted and returned. 
\\The reference value is given in mV.
\p example: \textit{adc 2 5000}
\\The command will return a value from 0.00V to 5.00 volt if no erros are encounter. 
\subsection{UART}
The UART program is made for UART2, so there is no need to specify the number when calling the program. The program takes the arguments: \textit{io}, \textit{read} and \textit{write}. When using the argument \textit{io}, the UART is configured to an 8n1, 19200 baudrate setup. The \textit{read} call returns a value from the UART input buffer if it is not empty, if it is empty, it goes to sleep and waits for a character to apear before it returns. When using \textit{write} another argument shall be sent to defining what character should be sent.
\subsection{PORT}
The PORT program controls the five power switch pins (see table \ref{tab:gpio_table}. The program takes two arguments, a number and a command. The number can be be \textit{0}, \textit{1} or \textit{2} while the command can be either: \textit{in}, \textit{out} or \textit{off}.
\p each port number corresponds to an power switch, where number 0 controls the grid and number 1 and 2 controls each of their modules. 
\p If argument is \textit{in}, the device routes energy from the module to the power line. If the argument is \textit{out}, energy is routed from the power line to the module connected. If \textit{off}, energy flow in the switch is disabled.