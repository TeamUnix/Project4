\chapter{Hardware design}
\section{Introduction}
\writer{NAME}
%

\section{Power line and switch mode}
\writer{Dennis Madsen}
%

\section{Power switch}
\writer{Paulo Fontes}
%

\section{Interface}
\writer{Dennis Madsen}
%

\section{Results}
\writer{NAME}
%

%The artefacts from the EUDP phases are listed in the report under “Results”

%Do not just copy and paste from EUDP process description, since the readers do not want to read the same twice. In the result section it is possible to refer to documents in the EUDP process documentation, and the EUDP documentation it is possible to refer to documents in the result section, but do not have the same text both places.

%In EPRO3, this may be the artefacts from PreProject and Launch, and the artefacts from Realisation and PostProject in EPRO4

%Since supervisor or examiner may not be familiar with these artefacts, it is important to include a brief description of what the terms cover

%PreProject is about finding out what needs to be done. The PreProject artefacts are therefore Rich Picture, Storytelling, an overall Use Case Diagram, Stakeholder Analysis, System Definition and PreContracting

%Launch is about finding out how. The Launch artefacts may be Exact Requirements, Technical Platform, Contracting, an overall Class Diagram as well as an overall Use Case Diagram

%Realisation is about realising the requirements for Launch. Thus, the artefacts from the Realisation phase are the Deployment artefacts from each timebox. However, it is important that each artefact is written cohesively in the timeboxes to provide a more continuous process in addition to keeping better track of the students’ work

%The PostProject must include all artefacts

%The remaining artefacts from PreProject, Launch and Realisation must be included in “Appendices”.